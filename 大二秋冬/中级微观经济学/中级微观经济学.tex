\documentclass[lang=cn,10pt]{elegantbook}
\usepackage{ctex}
\title{中级微观经济学(拔尖班)笔记}
\subtitle{2024-2025学年秋冬学期}

\author{Yishu Jiang}
\institute{School of Economics,Zhejiang University}
\date{\today}

\extrainfo{Talk is cheap,show me your solution!}

\setcounter{tocdepth}{3}

\logo{ZJU.png}
\cover{Cover.png}

% 本文档命令
\usepackage{array}
\newcommand{\ccr}[1]{\makecell{{\color{#1}\rule{1cm}{1cm}}}}

% 修改标题页的橙色带
% \definecolor{customcolor}{RGB}{32,178,170}
% \colorlet{coverlinecolor}{customcolor}

\setcounter{tocdepth}{2}
\begin{document}

\maketitle
\frontmatter
\chapter*{前言}
\markboth{Introduction}{前言}
{\fangsong 
    本笔记是2024-2025学年秋冬学期面向经济学拔尖班开设的中级微观经济学的学习笔记,主要使用的教材为范里安的《微观经济学 :现代观点》,本人使用范里安的《微观经济分析》、MWG的《微观经济理论》和平新乔的《微观经济学十八讲》作为参考书,授课老师为汪淼军.由于本人初次学习中微且水平能力有限,笔记中有所疏漏处,恳请指正.

    基本的授课思路是先介绍基本假设下的消费者理论和生产者理论, 接着从局部均衡到一般均衡, 得到福利经济学第一第二定理, 以及对市场均衡做进一步分析,再破除假设研究实际问题; 更加确切地说, 微观经济学更应当称为“价格理论”, 主要研究“看不见的手”在资源配置中的作用.

    这份笔记的思路基本按照汪老师的上课思路展开, 但在细节上会补充一些课上没有处理的特别好的技术细节, 力求让笔记阅读起来更加流畅, 因此会涉及一些高微的内容做补充 ,在初学时遇到这些内容如果感到困难的话可以先跳过.
}
\newpage

\tableofcontents

\mainmatter
\chapter{预备数学知识}
\section{最优化的数学工具:拉格朗日方法与哈密尔顿方法}
拉格朗日方法(一般用在静态的处理中)
$$\max_{\mathbf{x}\in\mathbb{R}^n}f(\mathbf{x})$$
$${\rm s.t.} \ \ g(\mathbf{x})\leq 0$$
构造拉格朗日函数$L(\mathbf{x};\lambda) = f(\mathbf{x})-\lambda g(x)$(经济学中一般喜欢让$\lambda>0$, 在后面有相应的经济学含义).\\
一阶条件$FOC$:$\frac{\partial L}{\partial x_i}=0$且$g(\mathbf{x})=0$.当然,解出来的结果并不一定就是最值, 还需要验证二阶条件; 另外, 拉格朗日方法只能给出内点解, 对于二阶条件和角点解的处理, 后续会介绍KKT条件.

哈密尔顿方法(往往和动态最优化相关)

\section{包络定理}
考虑一个带参数的最优化问题(这里假设$f$性质足够好):
$$\max_{\mathbf{x}\in\mathbb{R}^n}f(\mathbf{x};a)=:v(a)$$
记$\mathbf{x}^*\in\arg \max_{\mathbf{x}\in\mathbb{R}^n}f(\mathbf{x};a)$, 两边同时对$a$求导,有:
$$\frac{{\rm d}v(a)}{{\rm d}a}=\frac{\partial f}{\partial x_1^{*}}\frac{\partial x_{1}^{*}}{\partial a}+\cdots+\frac{\partial f}{\partial x_{n}^{*}}\frac{\partial x_{n}^{*}}{\partial a}+\frac{\partial f}{\partial a}$$
又根据最值的一阶条件:$\frac{\partial f}{\partial x_i}=0,i=1,2,\cdots,n$, 故有$$\frac{{\rm d}v(a)}{{\rm d}a}=\frac{\partial f}{\partial a}$$
包络定理在我们只关注参数对最值的影响时, 直接绕开了求解最值的过程, 对比较静态分析的化简有很大帮助.
\section{凹函数与拟凹函数}
$f:\mathbb{R}^n\to\mathbb{R}$为凹函数等价于$tf(x_1)+(1-t)f(x_2)\leq f(tx_1+(1-t)x_2),\forall t\in (0,1)$; 一般会假设$f''(\cdot)\leq 0$.
$f:\mathbb{R}^n\to\mathbb{R}$为拟凹函数等价于$f(tx_1+(1-t)x_2)\geq \min\{f(x_1),f(x_2)\},\forall t\in (0,1)$.
\chapter{消费者行为}
对于消费者, 我们要刻画其需求——构成需求需要三个元素:可选择的商品组合、价格以及消费者的收入.

基本知识框架 :
\begin{enumerate}
    \item 偏好关系与效用函数
    \item 消费集、预算集
    \item 约束下的最优化问题;无差异曲线、Walras需求函数、效用最大化问题、支出最小化问题、间接效用函数、花费函数、对偶关系.
\end{enumerate}
\newpage
\section{消费集和预算约束}

\subsection{消费集}
首先我们讨论消费者选择的对象: 商品.不考虑任何经济上的约束, 对于可供消费者选择的商品, 假设共有$n$种.用一个$n$维向量$x=(x_1,\cdots,x_n)$表示一个消费选择, 称作\textbf{消费束}, 所有消费束所构成的集合$X\subset \mathbb{R}^n_{+}$称作\textbf{消费集}.

消费集通常符合以下假设 :
\begin{enumerate}
    \item $X\subset \mathbb{R}_{+}^{n}$, 即消费数量非负(“减少一些东西”可以用绝对值替代).
    \item $0\in X$.
    \item $X$是\textbf{闭集}, 指$\partial X\subset X$(或者$\forall x\in X$,$\exists \{x_n\}\subset X,\mathrm{s.t.}\ x_n\to x,as\ n\to\infty$).换言之 :消费集中的任意一消费束都可以由一列消费束进行逼近,即消费集允许“极限行为”“擦边球行为”.
    \item $X$是\textbf{凸集}, 指$\forall x^1,x^2\in X,\forall \lambda\in [0,1],\lambda x^1+(1-\lambda)x^2\in X$, 换言之 :\textbf{消费集中任意两点连线上的所有点都在消费集内部},即可以通过连续的调整, 实现一种消费向另一种消费的过渡.
    \item $X$无上界.\textbf{消费集仅仅表示消费者客观上可选择的商品组合(也就是在刻画消费者的欲望,是允许无穷大的), 不考虑消费是否能够实现}.
\end{enumerate}
\subsection{预算约束}
进一步地, 我们引入经济约束——商品的价格和人们的收入限制了消费.设价格向量$p=(p_1,\cdots,p_n)$;设消费这的预算为$m$, 则有$p\cdot x\leq m$.在这一限制条件下, 所有可行的消费品集合为$\{x\in X\mid p\cdot x\leq m\}$, 它是消费集的子集——消费者能负担得起的所有消费束的集合, 也就是\textbf{预算集}.
当然, 预算集能表示成$\{x\in X\mid p\cdot x\leq m\}$, 得建立在\textbf{市场完备性}(所有商品的价格都是公开、透明的)的基础上.考虑最简单的情况, 价格$p$不变, 这种最简单的预算集被称为Walrasian运算集, 这建立在\textbf{价格接受}假设上, 仅当单个消费者的需求占总需求的占比很小时才成立.
此外, 还有一些因素也会影响消费者选择的可行域 :比如资源约束、分配方式(配额等).
预算集的边界$\{x\in X\mid p\cdot x=m\}$是$n$为空间中的$n-1$为超平面, 称为\textbf{预算超平面}.可以看出, 价格向量$p=(p_1,\cdots,p_n)$与预算超平面正交.
\newpage
\section{偏好关系和效用函数}
\section{偏好与选择}
接下来我们需要刻画消费者的选择行为本身. 微观经济学假定人是“理性”的, 反映在偏好上, 也就是说: 对于一个理性的消费者而言, 所有消费束都是可比较的并且这种比较是完备的.
\begin{definition}
偏好关系$\succ$是消费集$X$上的一个二元关系, $\forall x,y\in X,x\succeq y$当且仅当“$X$至少和$y$一样好”.
\end{definition}
由此引申出另两种二元关系:
\begin{definition}[严格偏好关系]
    $x\succ y$当且仅当$x\succeq y$但$y\succeq x$不成立.
\end{definition}
\begin{definition}[无差异关系]
    $x\sim y$当且仅当$x\succeq y$且$y\succeq x$.
\end{definition}
\subsection{特殊效用函数的计算事例}
以下均假设预算约束为$\sum_{i=1}^{n}p_ix_i\leq m$.

首先介绍CES效用函数(替代弹性恒定Constant Elasticity Substitution).
\begin{example}{\textbf{(CES效用函数)}}{}
    $$u(\textbf{x})=\left(\sum_{i=1}^{n}x_{i}^{\frac{\sigma-1}{\sigma}}\right)^{\frac{\sigma}{\sigma-1}}$$
\end{example}
边际效用$MU_{x_i}=x_i^{-\frac{1}{\sigma}}\left(\sum_{i=1}^{n}x_{i}^{\frac{\sigma-1}{\sigma}}\right)^{\frac{1}{\sigma-1}}$.

边际替代率$MRS_{x_i,x_j}=\frac{MU_{x_i}}{MU_{x_j}}=\frac{x_j}{x_i}^{\frac{1}{\sigma}}$

p;
\newpage
\section{消费者的最优选择}
在给定的约束下, 理性的消费者会选择自己最喜欢的商品组合.这样的最优化结果有两种刻画指标 :给定支付能力(收入), 获取最大效用——效用最大化问题(Utility Maximizing Problem,UMP);给定效用, 使用最小的支出——支出最小化问题(Expenditure Minimizing Problem,EMP).这两种问题在最优化问题的意义上且具有\textbf{对偶性}.

\subsection{效用最大化问题(UMP)与支出最小化问题(EMP)}
在理性人和价格接受者的假设下, 消费者的最优选择
\newpage

\subsection{对偶性: UMP与EMP的联系}
\newpage

\section{基于最优选择的进一步分析}
\newpage

\subsection{收入效应与替代效应}
\subsection{收入效应与替代效应的图示}
\subsection{Slustky方程}
\begin{theorem}{Slustky方程}
    设$x(p,y)$为马歇尔需求,$u$为消费者在价格$p$与收入$y$下的效用水平, 则有
    $$\underbrace{\frac{\partial x_i(p,y)}{\partial p_j}}_{TE} =\underbrace{\frac{\partial x_i^h(p,u)}{\partial p_j}}_{SE} \underbrace{-x_j(p,y)\frac{\partial x_i(p,y)}{\partial y}}_{IE} $$
    称为Slustky方程, 表示价格$p_j$为$x_i$消费量的总效应(TE)等于替代效应(SE)与收入效应(IE)之和.
\end{theorem}
\begin{proof}
    (\textbf{事实上,SLustky方程是UMP与EMP对偶关系的一个推论}).当$y=e(p,u^*)$时, 有$x_i^h(p,u^*)\equiv x_i(p,y)$.\\
    两边对$p_j$求偏导,得$\frac{\partial x_i^h}{\partial p_j}=\frac{\partial x_i}{\partial p_j}+\frac{\partial x_i}{\partial y}\frac{\partial y}{\partial p_j}$
    又$y=e(p,u)$, 由Shephard引理, 有$\frac{\partial y}{\partial p_j}=\frac{\partial e(p,u)}{\partial p_j}=x_j^h(p,u)=x_j$
\end{proof}
\newpage

\subsection{福利分析}
\newpage

\subsection{加总和需求}
\newpage

\section{其他问题}
\newpage

\subsection{不确定性下的选择}
\newpage

\subsection{跨期选择}
这一部分内容在中级宏观经济学中也有讲到(参见中宏里的两期模型).
\newpage

\chapter{生产者行为}
\section{技术}
\section{生产者的最优选择}
\subsection{利润最大化与利润函数}
\subsection{成本最小化与成本函数}
\subsection{对偶}
\section{短期生产分析}
区分生产的短期和长期, 看的是是否所有要去投入都可变.短期生产中部分要素投入固定, 衡量生产的主要指标是\textbf{边际报酬}.
\section{长期生产分析}
在长期生产中, 各种要素投入都可变, 衡量生产的主要指标是\textbf{规模报酬}.
\chapter{一般均衡理论}

\chapter{博弈论基础}
(写不动, 后边会专门写有关博弈论、机制设计、信息经济学的学习笔记).
\chapter{不完全竞争市场——市场结构分析}
\section{垄断与垄断行为分析}
\section{寡头与寡头博弈模型}
\chapter{市场失灵}
\section{公共物品}
\section{外部性}

\chapter{回忆卷整理}
{\kaishu 本部分为2024-2025学年小测和历年卷整理, 请勿外传.}
\section{第一次小测}
{\fangsong 第1题为必做题, 2-8题中选6题作答, 每题2分, 限时50分钟.}
\textcolor{red}{小测题基本来自平时讲过的例题, 有改编, 难度不大但请限时完成——尤其确保计算的准确性和熟练性, 50分钟做这样一张卷子时间相对紧张.}

\noindent 1. 判断题\\
(1).在其他条件不变时, 一种商品的价格上升最终导致其需求量增加, 则说明该商品价格变化产生的替代效应可能为正.\\
(2).如果收入的补偿变化越大, 则受政策影响的个体效用减少的越多; 如果收入的等价变化越大, 则受政策影响的个体效用减少的越多.\\
(3).如果所有生产要素按照相同比例变化至其原来的0.5倍, 而对应的产量也变化为原来的0.4倍, 则该生产技术具有规模报酬递增的特征.\\ 
(4).在完全竞争市场中, 自由进入行业必然导致企业的长期利润为零. 因此, 产品的市场价格必然是由生产产品的要素价格决定的.\\

\noindent 2. 某消费者消费三种商品, 效用函数为$u(x_1,x_2,x_3)=x_1+8(x_2+x_3)^\frac{1}{2}$, 商品$x_1,x_2,x_3$的价格分别为$4,4,8$, 消费者的收入水平为$m=120$.
求消费者对这三种商品的最优消费量.


\noindent 3. 在一个完全竞争市场中, 企业的生产函数为$f(x_1,x_2)=2x_1^{\frac{1}{2}}+2x_2^{\frac{1}{2}}$, 两种生产要素$x_1,x_2$对应的市场也是完全竞争的, 
且其市场价格分别为$4$和$4$. 若短期内要素$x_1=1$, 求出企业的短期供给函数.


\noindent 4. 企业有两家生产函数, 两家工厂的生产函数分别是$f(x_1,x_2)=\frac{1}{4}x_1^\frac{1}{2}x_2^\frac{1}{2}$和$f(x_1,x_2)=2x_1^\frac{1}{4}x_2^\frac{1}{4}$,
两种竞争性生产要素的市场价格分别是$1$和$4$. 当产品市场是竞争的且价格为$8$时, 求出企业两家工厂的最优产量.

\noindent 5. 间接效用函数$V(\mathbf{p},m)=u(\mathbf{x}(\mathbf{p},m))$, 其中$\mathbf{x}(\mathbf{p},m)$是在给定价格$\mathbf{p}$和收入$m$时使效用最大化的消费束.
证明: 对于$t>1$, 有$V(t\mathbf{p},tm)=V(\mathbf{p},m)$.\textcolor{red}{注意这道题没有可微的条件!}


\noindent 6. 在完全竞争市场中, 所有企业的成本函数都为$c(q)=q^2+20q+100$, 其中$q$为对应企业的产量. 汽车的需求市场由$500$个消费者组成, 其效用函数为$u=10x_1-\frac{1}{2}x_1^2+x_2$, 
$x_1$表示汽车, $x_2$由一个竞争性市场供给, 市场价格为$10$. 厂商进入或者退出不改变任何一家汽车企业的成本函数. 求出长期均衡时市场上存在的汽车厂商数目.


\noindent 7. 消费者初始财富为$W_0$, 存在$\pi$的概率可能遭受$L$的损失. 现在消费者决定购买$K$单位保险, 事前需要支付保险费用$\gamma \cdot K$, 在遭受损失时可获得赔偿$K$.
假设消费者效用函数$u(W)$满足: $u'(\cdot)>0,u''(\cdot)<0,u'''(\cdot)<0$, 其中$W$为消费者某一状态的财富水平. 现假设存在内点解和保险费率$\gamma>\pi$, 证明最优保险额度满足:
最优保险额度$K$是保险费率$\gamma$的单调递减函数.


\noindent 8. 在完全竞争的行业中, 某企业的生产函数$f(x_1,\cdots,x_n)$是严格递增的凹函数, 其生产要素$x_i,i=1,\cdots,n$都来自完全竞争市场. 
假设任意要素$x_i$对应的市场价格为$w_i$且满足$\frac{\partial^2 f}{\partial x_i\partial x_j}=\frac{\partial^2 f}{\partial x_j\partial x_i}$.
证明: 企业利润最大化满足:
$$\frac{\partial x_i}{\partial w_j}=\frac{\partial x_j}{\partial w_i}$$

\end{document}
