\documentclass[lang=cn,10pt]{elegantbook}
\usepackage{ctex}
\title{中级微观经济学(拔尖班)笔记}
\subtitle{2024-2025学年秋冬学期}

\author{Yishu Jiang}
\institute{School of Economics,Zhejiang University}
\date{\today}

\extrainfo{Talk is cheap,show me your solution!}

\setcounter{tocdepth}{3}

\logo{ZJU.png}
\cover{Cover.png}

% 本文档命令
\usepackage{array}
\newcommand{\ccr}[1]{\makecell{{\color{#1}\rule{1cm}{1cm}}}}

% 修改标题页的橙色带
% \definecolor{customcolor}{RGB}{32,178,170}
% \colorlet{coverlinecolor}{customcolor}

\setcounter{tocdepth}{2}
\begin{document}

\maketitle
\frontmatter
\chapter*{前言}
\markboth{Introduction}{前言}
{\fangsong 
    本笔记是2024-2025学年秋冬学期面向经济学拔尖班开设的中级微观经济学的学习笔记,主要使用的教材为范里安的《微观经济学 :现代观点》,本人使用范里安的《微观经济分析》、MWG的《微观经济理论》和平新乔的《微观经济学十八讲》作为参考书,授课老师为汪淼军.由于本人初次学习中微且水平能力有限,笔记中有所疏漏处,恳请指正.

    基本的授课思路是先介绍基本假设下的消费者理论和生产者理论, 接着从局部均衡到一般均衡, 得到福利经济学第一第二定理, 以及对市场均衡做进一步分析,再破除假设研究实际问题; 更加确切地说, 微观经济学更应当称为“价格理论”, 主要研究“看不见的手”在资源配置中的作用.

    这份笔记的思路基本按照汪老师的上课思路展开, 但在细节上会补充一些课上没有处理的特别好的技术细节, 力求让笔记阅读起来更加流畅, 因此会涉及一些高微的内容做补充 ,在初学时遇到这些内容如果感到困难的话可以先跳过.
}
\newpage

\tableofcontents

\mainmatter
\chapter{消费者行为}
对于消费者, 我们要刻画其需求——构成需求需要三个元素:可选择的商品组合、价格以及消费者的收入.

基本知识框架 :
\begin{enumerate}
    \item 偏好关系与效用函数
    \item 消费集、预算集
    \item 约束下的最优化问题;无差异曲线、Walras需求函数、效用最大化问题、支出最小化问题、间接效用函数、花费函数、对偶关系.
\end{enumerate}
\newpage
\section{消费集和预算约束}

\subsection{消费集}
首先我们讨论消费者选择的对象: 商品.不考虑任何经济上的约束, 对于可供消费者选择的商品, 假设共有$n$种.用一个$n$维向量$x=(x_1,\cdots,x_n)$表示一个消费选择, 称作\textbf{消费束}, 所有消费束所构成的集合$X\subset \mathbb{R}^n_{+}$称作\textbf{消费集}.

消费集通常符合以下假设 :
\begin{enumerate}
    \item $X\subset \mathbb{R}_{+}^{n}$, 即消费数量非负(“减少一些东西”可以用绝对值替代).
    \item $0\in X$.
    \item $X$是\textbf{闭集}, 指$\partial X\subset X$(或者$\forall x\in X$,$\exists \{x_n\}\subset X,\mathrm{s.t.}\ x_n\to x,as\ n\to\infty$).换言之 :消费集中的任意一消费束都可以由一列消费束进行逼近,即消费集允许“极限行为”“擦边球行为”.
    \item $X$是\textbf{凸集}, 指$\forall x^1,x^2\in X,\forall \lambda\in [0,1],\lambda x^1+(1-\lambda)x^2\in X$, 换言之 :\textbf{消费集中任意两点连线上的所有点都在消费集内部},即可以通过连续的调整, 实现一种消费向另一种消费的过渡.
    \item $X$无上界.\textbf{消费集仅仅表示消费者客观上可选择的商品组合(也就是在刻画消费者的欲望,是允许无穷大的), 不考虑消费是否能够实现}.
\end{enumerate}
\subsection{预算约束}
进一步地, 我们引入经济约束——商品的价格和人们的收入限制了消费.设价格向量$p=(p_1,\cdots,p_n)$;设消费这的预算为$m$, 则有$p\cdot x\leq m$.在这一限制条件下, 所有可行的消费品集合为$\{x\in X\mid p\cdot x\leq m\}$, 它是消费集的子集——消费者能负担得起的所有消费束的集合, 也就是\textbf{预算集}.
当然, 预算集能表示成$\{x\in X\mid p\cdot x\leq m\}$, 得建立在\textbf{市场完备性}(所有商品的价格都是公开、透明的)的基础上.考虑最简单的情况, 价格$p$不变, 这种最简单的预算集被称为Walrasian运算集, 这建立在\textbf{价格接受}假设上, 仅当单个消费者的需求占总需求的占比很小时才成立.
此外, 还有一些因素也会影响消费者选择的可行域 :比如资源约束、分配方式(配额等).
预算集的边界$\{x\in X\mid p\cdot x=m\}$是$n$为空间中的$n-1$为超平面, 称为\textbf{预算超平面}.可以看出, 价格向量$p=(p_1,\cdots,p_n)$与预算超平面正交.
\newpage
\section{偏好关系和效用函数}
\section{偏好与选择}
接下来我们需要刻画消费者的选择行为本身. 微观经济学假定人是“理性”的, 反映在偏好上, 也就是说: 对于一个理性的消费者而言, 所有消费束都是可比较的并且这种比较是完备的.
\begin{definition}
偏好关系$\succ$是消费集$X$上的一个二元关系, $\forall x,y\in X,x\succeq y$当且仅当“$X$至少和$y$一样好”.
\end{definition}
由此引申出另两种二元关系:
\begin{definition}[严格偏好关系]
    $x\succ y$当且仅当$x\succeq y$但$y\succeq x$不成立.
\end{definition}
\begin{definition}[无差异关系]
    $x\sim y$当且仅当$x\succeq y$且$y\succeq x$.
\end{definition}
\subsection{特殊效用函数的计算事例}
以下均假设预算约束为$\sum_{i=1}^{n}p_ix_i\leq m$.

首先介绍CES效用函数(替代弹性恒定Constant Elasticity Substitution).
\begin{example}{\textbf{(CES效用函数)}}{}
    $$u(\textbf{x})=\left(\sum_{i=1}^{n}x_{i}^{\frac{\sigma-1}{\sigma}}\right)^{\frac{\sigma}{\sigma-1}}$$
\end{example}
边际效用$MU_{x_i}=x_i^{-\frac{1}{\sigma}}\left(\sum_{i=1}^{n}x_{i}^{\frac{\sigma-1}{\sigma}}\right)^{\frac{1}{\sigma-1}}$.

边际替代率$MRS_{x_i,x_j}=\frac{MU_{x_i}}{MU_{x_j}}=\frac{x_j}{x_i}^{\frac{1}{\sigma}}$

p;
\newpage
\section{消费者的最优选择}
在给定的约束下, 理性的消费者会选择自己最喜欢的商品组合.这样的最优化结果有两种刻画指标 :给定支付能力(收入), 获取最大效用——效用最大化问题(Utility Maximizing Problem,UMP);给定效用, 使用最小的支出——支出最小化问题(Expenditure Minimizing Problem,EMP).这两种问题在最优化问题的意义上且具有\textbf{对偶性}.

\subsection{效用最大化问题(UMP)}
\newpage

\subsection{支出最小化问题(EMP)}
\newpage

\subsection{对偶性: UMP与EMP的联系}
\newpage

\section{基于最优选择的进一步分析}
\newpage

\subsection{收入效应与替代效应}
\newpage

\subsection{福利分析}
\newpage

\subsection{加总和需求}
\newpage

\section{其他问题}
\newpage

\subsection{不确定性下的选择}
\newpage

\subsection{跨期选择}
这一部分内容在中级宏观经济学中也有讲到.
\newpage

\chapter{生产者行为}
\chapter{一般均衡理论}
\chapter{博弈论基础}
\chapter{市场结构分析}
\chapter{市场失灵}
\end{document}
