
\chapter{回忆卷整理}
{\kaishu 本部分为2024-2025学年小测和历年卷整理, 请勿外传.}
\section{第一次小测}
{\fangsong 第1题为必做题, 2-8题中选6题作答, 每题2分, 限时50分钟.}
\textcolor{red}{小测题基本来自平时讲过的例题, 有改编, 难度不大但请限时完成——尤其确保计算的准确性和熟练性, 50分钟做这样一张卷子时间相对紧张.}

\noindent 1. 判断题\\
(1).在其他条件不变时, 一种商品的价格上升最终导致其需求量增加, 则说明该商品价格变化产生的替代效应可能为正.\\
(2).如果收入的补偿变化越大, 则受政策影响的个体效用减少的越多; 如果收入的等价变化越大, 则受政策影响的个体效用减少的越多.\\
(3).如果所有生产要素按照相同比例变化至其原来的0.5倍, 而对应的产量也变化为原来的0.4倍, 则该生产技术具有规模报酬递增的特征.\\ 
(4).在完全竞争市场中, 自由进入行业必然导致企业的长期利润为零. 因此, 产品的市场价格必然是由生产产品的要素价格决定的.\\

\noindent 2. 某消费者消费三种商品, 效用函数为$u(x_1,x_2,x_3)=x_1+8(x_2+x_3)^\frac{1}{2}$, 商品$x_1,x_2,x_3$的价格分别为$4,4,8$, 消费者的收入水平为$m=120$.
求消费者对这三种商品的最优消费量.


\noindent 3. 在一个完全竞争市场中, 企业的生产函数为$f(x_1,x_2)=2x_1^{\frac{1}{2}}+2x_2^{\frac{1}{2}}$, 两种生产要素$x_1,x_2$对应的市场也是完全竞争的, 
且其市场价格分别为$4$和$4$. 若短期内要素$x_1=1$, 求出企业的短期供给函数.


\noindent 4. 企业有两家生产函数, 两家工厂的生产函数分别是$f(x_1,x_2)=\frac{1}{4}x_1^\frac{1}{2}x_2^\frac{1}{2}$和$f(x_1,x_2)=2x_1^\frac{1}{4}x_2^\frac{1}{4}$,
两种竞争性生产要素的市场价格分别是$1$和$4$. 当产品市场是竞争的且价格为$8$时, 求出企业两家工厂的最优产量.

\noindent 5. 间接效用函数$V(\mathbf{p},m)=u(\mathbf{x}(\mathbf{p},m))$, 其中$\mathbf{x}(\mathbf{p},m)$是在给定价格$\mathbf{p}$和收入$m$时使效用最大化的消费束.
证明: 对于$t>1$, 有$V(t\mathbf{p},tm)=V(\mathbf{p},m)$.\textcolor{red}{注意这道题没有可微的条件!}


\noindent 6. 在完全竞争市场中, 所有企业的成本函数都为$c(q)=q^2+20q+100$, 其中$q$为对应企业的产量. 汽车的需求市场由$500$个消费者组成, 其效用函数为$u=10x_1-\frac{1}{2}x_1^2+x_2$, 
$x_1$表示汽车, $x_2$由一个竞争性市场供给, 市场价格为$10$. 厂商进入或者退出不改变任何一家汽车企业的成本函数. 求出长期均衡时市场上存在的汽车厂商数目.


\noindent 7. 消费者初始财富为$W_0$, 存在$\pi$的概率可能遭受$L$的损失. 现在消费者决定购买$K$单位保险, 事前需要支付保险费用$\gamma \cdot K$, 在遭受损失时可获得赔偿$K$.
假设消费者效用函数$u(W)$满足: $u'(\cdot)>0,u''(\cdot)<0,u'''(\cdot)<0$, 其中$W$为消费者某一状态的财富水平. 现假设存在内点解和保险费率$\gamma>\pi$, 证明最优保险额度满足:
最优保险额度$K$是保险费率$\gamma$的单调递减函数.


\noindent 8. 在完全竞争的行业中, 某企业的生产函数$f(x_1,\cdots,x_n)$是严格递增的凹函数, 其生产要素$x_i,i=1,\cdots,n$都来自完全竞争市场. 
假设任意要素$x_i$对应的市场价格为$w_i$且满足$\frac{\partial^2 f}{\partial x_i\partial x_j}=\frac{\partial^2 f}{\partial x_j\partial x_i}$.
证明: 企业利润最大化满足:
$$\frac{\partial x_i}{\partial w_j}=\frac{\partial x_j}{\partial w_i}$$
