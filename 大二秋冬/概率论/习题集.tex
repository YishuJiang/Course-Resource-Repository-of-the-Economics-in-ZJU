
\chapter{茆书上习题}
\chapter{吴昊讲义上习题}
\chapter{小测题}
\section{第一次小测}
\noindent 1.设有$n$次独立重复随机试验,每次出现0,1,2的概率分别是$p_0,p_1,p_2$,其中$p_0+p_1+p_2=1$,求$n$次试验中1和2都至少出现一次的概率.

\noindent 2.设$B_1,B_2,\cdots,B_n$为一系列相互独立的事件(即它们中任意有限个都是相互独立的).$P(B_n)=p_n,0<p_n<1,\sum_{n=1}^{\infty}p_n=\infty$,证明:
$$P\left(\bigcup_{n=m}^\infty B_n\right)=1,\forall m$$

\noindent 3.考虑进行$n$次独立试验,第$i$次试验成功的概率为$\frac{1}{2i+1}$,令$P_n$表示总的成果次数为奇数的概率.\\
(1).导出$P_{n-1}$和$P_n$的递推公式.\\
(2).导出$P_n$的表达式.

\noindent 4.假设$E,F$为任意两个事件,$P(F)>0$,证明:
$$P(E\mid E\cup F)\geq P(E\mid F)$$

\noindent 5.证明$\left|P(AB)-P(A)-P(B)\right|\leq\frac{1}{4}$

\noindent 6.某赌徒口袋里有一枚均匀硬币和一枚两面都是“正面”的硬币,他随机地摸出一枚,\\
(1).假设他抛掷这枚硬币后出现了正面,求这枚硬币是均匀硬币的概率.\\
(2).假设他抛掷这枚硬币$k$次出现了$k$次正面,求这枚硬币是均匀硬币的概率.
\newpage
\section{第二次小测}
\noindent 1.考虑$n$次独立重复抛掷一枚硬币,每次硬币正面朝上的概率为$p$.证明有偶数个正面朝上的概率为$\frac{1+(q-p)^n}{2}$.其中$q=1-p$.

\noindent 2.(1).设$X_1,X_2,\cdots,X_n$为独立同分布的标准正态随机变量, 证明:$$b_1X_1+b_2X_2+\cdots+b_nX_n\sim N(0,\sigma^2),\sigma^2=b^2_1+b^2_2+\cdots+b^2_n$$
(2).设$X_1,X_2,\cdots,X_n$为相互独立的$N(0,1)$变量,$b_1,b_2,\cdots,b_n$为实数,证明:$b_1X_1+\cdots+b_nX_n\sim N(0,\sigma^2)$,其中$\sigma^2=b_1^2+\cdots+b_n^2$.\\
(3).设$\xi_1,\xi_2,\cdots,\xi_6$为相互独立的$N(0,1)$变量.记
$$\eta=\frac{\xi_1\xi_4+\xi_2\xi_5+\xi_3\xi_6}{\sqrt{\xi^2_4+\xi_5^2+\xi_6^2}}$$
求$\eta$的分布.

\noindent 3.设$\xi$和$\eta$独立,$\eta\sim\gamma(\lambda,a+b),\xi\sim\beta(a,b)$,求$\xi\eta$的分布.

\noindent 4.设$X_1,\cdots,X_n$为相互独立同分布的随机变量,均服从参数为$\lambda>0$的指数分布,$N$服从参数为$\beta$的Possion分布,且与$X_1,X_2,\cdots$独立.记$X_0=0$,定义$T$如下:对每个$\omega$,$T(\omega)=\max_{0\leq k \leq N(\omega)}X_k(\omega)$.\\
(1).证明$T$是随机变量.\\
(2).求$T$的分布函数与密度函数.

\noindent 5.设$X_1,\cdots,X_n$为相互独立的标准正态随机变量,令$S_j=X_1+\cdots+X_j,1\leq j\leq n$.\\
(1).求在给定$S_j=x$的条件下,$S_n$的条件分布$(1\leq j <n)$\\
(2).求在给定$S_n=y$的条件下,$S_j$的条件分布$(1\leq j <n)$.
\newpage
\section{第三次小测}
\noindent 1.设$X$服从参数为$M,N$和$n$的超几何分布,即$\mathbb{P}=\frac{\binom{M}{k}\binom{N-M}{n-k}}{\binom{N}{n}},k=0,1,2,\cdots$,其中$n\leq M\leq N$.
求$EX$和$Var X$

\noindent 2.设$X$和$Y$为相互独立的标准正态随机变量,求$XY$的特征函数.

\noindent 3.设$X_1,\cdots,X_n,\cdots$是一列相互独立同分布的标准正态变量,$N_n$服从二项分布$B(n,p)(0<p<1)$并与$X_1,\cdots,X_n,\cdots$独立.证明 :
$$\xi_n=\frac{\sum_{k=1}^{N_n}X_k}{\sqrt{n}} \xrightarrow{d} N(0,p),n\to \infty$$

\noindent 4.$g(t)$为定义在$[0,+\infty)$上的单调递减非负连续函数,记$\int_{0}^{+\infty}g(t){\rm d}t=a,\int_{0}^{+\infty}tg(t){\rm d}t=b$,设$0<a,b<+\infty$.
设随机变量$X,Y$的联合密度函数为$p(x,y)=g(x^2+y^2),-\infty<x,y<+\infty$.\\
(1)求$a$的值.\\
(2)求随机变量$X,Y$的期望、方差.\\
(3)求$X,Y$的相关系数.\\
(4)设$X,Y$相互独立,证明$X,Y$均服从正态分布.

\chapter{历年卷}
\section{2023-2024概率论3.5学分zlx班}
\noindent 1. 证明:
$$P\left(\bigcup_{i=1}^nA_i\right)=\sum_{i=1}^nP(A_i)-\sum_{1\leq i<j\leq n}P(A_i\cap A_j)+\cdots+(-1)^nP(A_1\cap A_2\cap \cdots \cap A_n)$$

\noindent 2. 两批零件,第一种$n_1$个,寿命$X_1,\cdots,X_{n_1}\sim E(\lambda_1)$,第二种$n_2$个,寿命$Y_1,\cdots,Y_{n_2}\sim E(\lambda_2)$,有一个零件失效则失效.记$T$为失效时间.\\
(1).证明:$T\sim E(\lambda_1n_1+\lambda_2n_2)$;\\
(2).求第一种零件失效导致失效的概率.

\noindent 3. $X_1,\cdots,X_n$独立同分布$\sim P(\lambda)$.且有$$S=X_1+\cdots+X_n,\overline{X}=\frac{X_1+\cdots+X_n}{n},T=\frac{1}{n-1}\sum_{i=1}^{n}(x_i-\overline{x})^2$$
(1).计算$ET$;\\
(2).$S=s(s=0,1,\cdots)$时,证明$X_i\sim B(s,\frac{1}{n})$;\\
(3).计算$E(T\mid S)$.

\noindent 4. $p(x,y)=C(x-y)^2{\rm e}^{-\frac{1}{2}(x^2+y^2)}$.\\
(1).求$C$;\\
(2).求$P_X(x),P_Y(y)$;\\
(3).计算$r_{XY}$;\\
(4).证明$X+Y,X-Y$独立.

\noindent 5. $X_i$独立同分布,$E|X_1|<\infty,\mu=EX_1$,证明:$$\frac{S_n}{n}\xrightarrow{P}\mu$$

\noindent 6. $N_p$服从几何分布,参数为$p$,$X_i$独立同分布$\sim N(\mu,\sigma^2),Y_p=\sum_{k=1}^{N_p}X_k$.\\
(1).证明$Y_p$是随机变量;\\
(2).求$EY_p$;\\
(3).求$Var Y_p$;\\
(4).证明$Y_p$不是正态随机变量.

\noindent 7. (附加题)(1).证明$Y_p$是连续型随机变量;\\
(2).$p\to 0$,证明$pY_p$收敛到某个分布函数.
\newpage

\section{2022-2023概率论3.5学分zlx班}
\noindent 1. 设事件序列$\left\{A_n\right\}$相互独立,$P(A_n)=\frac{1}{n^2}$,求$P\left(\bigcup_{n=m}^{\infty}A_n\right)$.

\noindent 2. $\left\{\xi_k\right\}$为独立同分布的随机变量,$\xi_k\sim E(\lambda),S_k=\sum_{i=1}^k\xi_i$.\\
(1).求证$S_k$服从参数为$k$和$\lambda$的gamma分布.\\
(2).令$N=\max_{k}\left\{S_k\leq x\right\}$.求证:$N$服从参数为$\lambda_x$的Poisson分布.(提示:$P(N=k)=p(S_k\leq x,S_{k+1}>x)$)

\noindent 3. $X_1,X_2$为标准正态随机变量,$Y=\mathrm{e}^{X_1-X_2}$.\\
(1).求$EY,VarY$.\\
(2).求$Y$与$X_1$的协方差$Cov(Y,X_1)$以及相关系数$\beta$.\\
(3).设随机变量$Z=Y-\alpha X_1-\beta X_2$.求$\alpha,\beta$使得$Z$与$X_1$和$X_2$均不相关.

\noindent 4. $X,Y$为随机变量,密度函数
$$p(x,y)=c(x+y)\exp{-x^2-y^2}$$
(1).求证:$X+Y$与$X-Y$相互独立.\\
(2).求$c$的值.\\
(3).求$X+Y$与$X-Y$的密度函数.

\noindent 5. $\left\{\xi_n\right\}$为一列独立同分布随机变量,$E\xi_1=\mu,Var\xi_1=\sigma^2,T_n=\sum_{1\leq i<j\leq n}\xi_i\xi_j$.\\
(1).求$T_n$的期望、方差;\\
(2).求证存在常数$c$使得$\frac{T_n}{n^2}\xrightarrow{P}c$并求$c$的值.

\noindent 6. $\left\{\xi_i\right\}$为一列独立同分布随机变量,$E\xi_i=\mu$,$N_p$为服从几何分布的随机变量($P(N_p=k)=p(1-p)^k$).$Y_p=\sum_{i=1}^{N_p}\xi_i$.\\
(1).求证:$Y_p$为随机变量.\\
(2).求证:$p\to 0$时,$pY_p$依分布收敛,并求其收敛分布.
\newpage
\section{2021-2022概率论3.5学分zlx班}
\noindent 1. (1).$\left\{A_n\right\}$为独立事件列,$\sum_{n=1}^{\infty}P(A_n)=\infty$,求证:$P\left(\bigcup_{n=m}^{\infty}A_n\right)=1,\forall m\in \mathbb{N}_+$.\\
(2).$X,Y$为独立且服从几何分布的随机变量,参数为$p$,求证:$P(X=i\mid X+Y=n)=\frac{1}{n-1},i=1,2,\cdots$

\noindent 2. $S_n=\sum_{k=1}^n\sin^2(kU),U\sim U(0,2\pi)$.\\
(1).求$ES_n$.\\
(2).证明$\frac{S_n}{n}\xrightarrow{P}c$并求出$c$.

\noindent 3. $X,Y\sim N(0,0,1,1,r)$.\\
(1).求出$a$使得$Y$与$X-aY$不相关,并求出$X-aY$的分布函数.\\
(2).求$X^2$与$Y^2$的相关系数.

\noindent 4. $p(x,y)=C(x-y)^2\exp{\left\{-\frac{1}{2}(x^2+y^2)\right\}}$为$X,Y$联合密度.\\
(1).求$C$.\\
(2).求$EX,EY,VarX,VarY$.\\
(3).证明:$X+Y$与$X-Y$独立.\\
(4).求$X^2+Y^2$的密度.

\noindent 5. 针对$X,Y$为离散形式的情况,利用期望的定义证明:$$EXY=EX\cdot EY$$

\noindent 6. 已知$\xi_i\xrightarrow{P}\varepsilon(\lambda),E(\xi_1)=\frac{1}{\lambda},N_n\sim \mathcal{P}(n)$(参数为$n$的Poisson分布).记$\eta_n=\frac{\sum_{k=1}^{N_n}\xi_k}{n}$.\\
(1).证明$\eta_n$是随机变量,求出$\eta_n$的特征函数.\\
(2).$\sqrt{n}(\eta_n-a)\xrightarrow{d}N(0,b)$并求出对应的$a$和$b$.


