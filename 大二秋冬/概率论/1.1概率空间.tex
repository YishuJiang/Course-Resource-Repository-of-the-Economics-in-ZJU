\chapter{概率空间}
在讨论概率空间之前,首先我们需要用对集合的运算刻画事件——在实际情况中往往因为信息问题,我们只能知道其中一部分事件的行为,而需要刻画所有事件的概率,我们首先需要找到一个能对事件的运算封闭的集合.
\section{概率的公理化定义}
此处给出Jordan公式, 用数学归纳法证明. 后续引入示性函数后还有更加简洁的证法.
\begin{theorem}{Jordan公式}{}
    
\end{theorem}
\section{概率的连续性}
类似函数极限$\lim_{n\to\infty}f(x_n)=f(\lim_{n\to\infty}x_n)$,我们不禁要问,概率测度是否也具有这样良好的性质,即允许极限与概率测度换序.

从简单的事件列研究起,由于概率空间中的事件一定是"有界"的($\varnothing \subset A\subset \Omega$),参照数列极限,我们首先考虑单调序列:对于单调增的事件列$\{A_n\}_{n=1}^{\infty}$,有$A_1\subset A_2\subset \cdots$,则有$\bigcup_{i=1}^{\infty}A_i=\lim_{n\to\infty}A_n$;同理,对于单调减的事件列
$\{B_n\}_{n=1}^{\infty}$,指$B_1\supset B_2\supset \cdots$,则有$\bigcap_{i=1}^{\infty}B_i=\lim_{n\to\infty}B_n$.基于此,我们可以得到:
\begin{theorem}{概率的上、下连续性}{}
    对于单调增的事件列$\{A_n\}_{n=1}^{\infty}$,有$\lim_{n\to\infty}\mathbb{P}(A_n)=\mathbb{P}(\lim_{n\to\infty}A_n)=\mathbb{P}(\cup_{n=1}^{\infty}A_n)$,称\textbf{上连续性}.

    对于单调减的事件列$\{B_n\}_{n=1}^{\infty}$,有$\lim_{n\to\infty}\mathbb{P}(B_n)=\mathbb{P}(\lim_{n\to\infty}B_n)=\mathbb{P}(\cap_{n=1}^{\infty}B_n)$,称\textbf{下连续性}.
\end{theorem}
\begin{proof}
    首先考虑单调增的事件列的情形.我们希望能使用概率的“可列可加性”,因此需要把$\{A_n\}$转化成一列相互不相交的事件列.做如下变换 :
    $$B_k=A_k-A_{k-1},k=1,2,\cdots;A_0=\varnothing$$
    则$\{B_k\}$相互不相交且有$\cup_{i=1}^{\infty}{A_i}=\cup_{i=1}^{\infty}B_i$(类似于在Venn图里拆成"一圈一圈的饼").\\
    故$\mathbb{P}(\cup_{n=1}^{\infty}A_n)=\mathbb{P}(\cup_{n=1}^{\infty}B_n)=\sum_{n=1}^{\infty}\mathbb{P}(B_n)=\sum_{n=1}^{\infty}(\mathbb{P}(A_{n})-\mathbb{P}(A_{n-1}))=\lim_{n\to\infty}\mathbb{P}(A_n)$.\\
    而对于单调减的事件列,只需要取单调增情形时的余集,即可套用上面的过程得到结论.
\end{proof}

现在考虑一般的事件列$\{A_n\}$,在没有单调性的情形下,再借鉴数列极限中的上下极限的概念,我们定义事件列的上下极限.
\begin{definition}{事件列的上下极限}{}
    对于事件列$A_1,A_2,\cdots$
    设$\left\{\begin{aligned}C_n=\bigcup_{j=n}^{\infty}A_j\\D_n=\bigcap_{j=n}^{\infty}A_j\end{aligned}\right.$易知$C_n$单调递减,$D_n$单调递增,则极限存在.
   $$\lim_{n\to\infty}C_n=\bigcap_{n=1}^{\infty}\bigcup_{j=n}^{\infty}A_j=:\overline{\lim}_{n\to \infty}A_n(\sup_{n\geq 1}\inf_{m\geq n}A_m)$$ 
   $$\lim_{n\to\infty}D_n=\bigcup_{n=1}^{\infty}\bigcap_{j=n}^{\infty}A_j=:\underline{\lim}_{n\to\infty}A_n(\inf_{n\geq 1}\inf_{m\geq n}A_n)$$
\end{definition}
从定义中,可以看出:下极限是上极限的子集,即$\underline{\lim}_{n\to\infty}A_n\subset \overline{\lim}_{n\to \infty}A_n$.于是可以定义 :$\lim_{n\to\infty}A_n$存在当且仅当$\underline{\lim}_{n\to\infty}A_n=\overline{\lim}_{n\to \infty}A_n$\\
更进一步地理解,可以发现$\overline{\lim}_{n\to \infty}A_n$发生当且仅当有无穷个$A_j$发生,故而也被记作$\{A_n\ i.o.\}$(infinitely often);$\underline{\lim}_{n\to\infty}A_n$发生当且仅当最多有限个$A_j$不发生.

在此基础上,我们证明概率测度的连续性:$\exists\lim_{n\to\infty}A_n\Rightarrow\lim_{n\to\infty}\mathbb{P}(A_n)=\mathbb{P}(\lim_{n\to\infty}A_n)$.
\begin{lemma}{Fatou引理}{}
    $$\mathbb{P}(\lim_{n\to\infty}\inf A_n)=\mathbb{P}(\cup_{n=1}^{\infty}\cap_{m=n}^{\infty}A_m)=\lim_{n\to\infty}\mathbb{P}(\cap_{m=n}^{\infty}A_m)\leq\lim_{n\to\infty}\inf \mathbb{P}(A_n)$$
    $$\mathbb{P}(\lim_{n\to\infty}\sup A_n)=\mathbb{P}(\cap_{n=1}^{\infty}\cup_{m=n}^{\infty}A_m)=\lim_{n\to\infty}\mathbb{P}(\cup_{m=n}^{\infty}A_m)\geq\lim_{n\to\infty}\sup \mathbb{P}(A_n)$$
\end{lemma}
\noindent 要使$\lim_{n\to\infty}A_n$存在,则有$\lim_{n\to\infty}\inf A_n=\lim_{n\to\infty}\sup A_n=\lim_{n\to\infty}A_n$.由夹逼定理知$\lim_{n\to\infty}\mathbb{P}(A_n)=\mathbb{P}(\lim_{n\to\infty}A_n)$即证明了概率测度的连续性.

更进一步地,可以证明,概率测度的“可列可加性”等价于“有限可加性+连续性”.
\begin{proposition}
    设$\mathbb{P}:\mathcal{F}\to[0,1]$满足有限可加性,且对$\{A_n\}\subset \mathcal{F},A_n\searrow \varnothing$有$\mathbb{P}(A_n)\to 0$.
    则$\mathbb{P}$满足可列可加性.
\end{proposition}
\begin{proof}
    考虑事件列$\{B_n\}_{n=1}^{\infty}$满足$B_i\cap B_j=\varnothing,\forall i,j$且$B_n\searrow $.\\
    令$B=\sum_{i=1}^{\infty}B_i=\sum_{i=1}^{n}B_i+\sum_{i=n+1}^{\infty}B_i$,并设$C_n=\sum_{i=n+1}^{\infty}B_i$则$C_n\searrow\varnothing$.\\
    于是$\mathbb{P}(B)=\mathbb{P}(\sum_{i=1}^{n}B_i)+\mathbb{P}(C_n)\to\mathbb{P}(\sum_{i=1}^{n}B_i)\ as\ n\to\infty$.
\end{proof}
\section{Borel-Canteli引理}
重新回顾一下集合列的上下极限.
\begin{definition}{事件列的上下极限}{}
    对于事件列$A_1,A_2,\cdots$
    设$\left\{\begin{aligned}C_n=\bigcup_{j=n}^{\infty}A_j\\D_n=\bigcap_{j=n}^{\infty}A_j\end{aligned}\right.$易知$C_n$单调递减,$D_n$单调递增,则极限存在.
   $$\lim_{n\to\infty}C_n=\bigcap_{n=1}^{\infty}\bigcup_{j=n}^{\infty}A_j=:\overline{\lim}_{n\to \infty}A_n$$ 
   $$\lim_{n\to\infty}D_n=\bigcup_{n=1}^{\infty}\bigcap_{j=n}^{\infty}A_j=:\underline{\lim}_{n\to\infty}A_n$$
\end{definition}

在定义了事件列上下极限的基础上,引入Borel-Canteli引理.
\begin{theorem}{Borel-Canteli引理}
    对于事件列$A_1,A_2,\cdots$\\
    (1):若$\sum_{j=1}^{\infty}\mathbb{P}(A_j)<\infty$,则$\mathbb{P}(A_n \ i.o.)=0$;\\
    (2):若事件列中事件相互独立,$\sum_{j=1}^{\infty}\mathbb{P}(A_j)=\infty$,则$\mathbb{P}(A_n \ i.o.)=1$.
\end{theorem}
{\kaishu
\noindent 思路:\\
(1).$\mathbb{P}(A_n\ i.o.)=\lim_{n\to\infty}\mathbb{P}(\bigcup_{j=n}^{\infty}A_j)$,考虑到条件给的是事件列概率的和,故而我们可以想到通过次可列可加性,将事件列并的概率放缩成概率列的部分和,再通过正项级数收敛的性质进行估计.\\
(2).$\mathbb{P}(A_n\ i.o.)=\lim_{n\to\infty}\mathbb{P}(\bigcup_{j=n}^{\infty}A_j)$,对$\mathbb{P}(\bigcup_{j=n}^{\infty}A_j)$进行估计,由于要使用到事件相互独立的条件,我们可以通过De-Morgan定律将并改写为交,即得到
$\mathbb{P}(\bigcup_{j=n}^{\infty}A_j)=\mathbb{P}(\overline{\overline{\bigcup_{j=n}^{\infty}A_j}})=1-\mathbb{P}(\bigcap_{j=n}^{\infty}\overline{A_j})=1-\prod_{j=n}^{\infty}\mathbb{P}(\overline{A_j})=1-\prod_{j=n}^{\infty}(1-\mathbb{P}(A_j))$.
到此为止,我们用完了事件独立性的条件,接下来的目标就是再进行估计,把概率和的条件用上,就需要把乘积转换成和的形式,考虑如下放缩$1-x\leq {\rm e}^{-x},x\in[0,1]$,于是上式可以改写成一个指数上带求和的形式,从而得到收敛.
}
\begin{proof}
    (1):由$\mathbb{P}(A_n\ i.o.)=\lim_{n\to\infty}\mathbb{P}(\bigcup_{j=n}^{\infty}A_j)$,又$\mathbb{P}(\bigcup_{j=n}^{\infty}A_j)\leq \sum_{j=n}^{\infty}\mathbb{P}(A_j)$,
    得$\mathbb{P}(A_n\ i.o.)\leq \lim_{n\to\infty}\sum_{j=n}^{\infty}\mathbb{P}(A_j)$.而$\sum_{j=1}^{\infty}\mathbb{P}(A_j)<\infty$,则根据级数收敛的Cauchy准则知$\lim_{j=n}^{\infty}\mathbb{P}(A_j)=0$.
    故$\mathbb{P}(A_n\ i.o.)=0$.

    (2):由$\mathbb{P}(A_n\ i.o.)=\lim_{n\to\infty}\mathbb{P}(\bigcup_{j=n}^{\infty}A_j)$,又$$\mathbb{P}(\bigcup_{j=n}^{\infty}A_j)=\mathbb{P}(\overline{\overline{\bigcup_{j=n}^{\infty}A_j}})=1-\mathbb{P}(\bigcap_{j=n}^{\infty}\overline{A_j})=1-\prod_{j=n}^{\infty}\mathbb{P}(\overline{A_j})=1-\prod_{j=n}^{\infty}(1-\mathbb{P}(A_j))$$
    考虑$1-x\leq{\rm e}^{-x},0\leq x\leq 1$,有$\mathbb{P}(\bigcup_{j=n}^{\infty}A_j)\leq 1-\prod_{j=n}^{\infty}{\rm e}^{-\mathbb{P}(A_j)}=1-{\rm e}^{-\sum_{j=n}^{\infty}\mathbb{P}(A_j)}$.
    由于$\sum_{j=1}^{\infty}\mathbb{P}(A_j)=\infty$,则$-\sum_{j=n}^{\infty}\mathbb{P}(A_j)=-\infty$,故可以得到$\mathbb{P}(\bigcup_{j=n}^{\infty}A_j)=1$,于是有$\mathbb{P}(A_n\ i.o.)=1$.
\end{proof}
\noindent 注:
\begin{enumerate}
    \item 在事件列中事件是相互独立时,$\mathbb{P}(A_n\ i.o.)$只能取0或1,即对相互独立的事件列,其中有无穷个事件发生的概率要么为0要么为1.这一结论被称为\textbf{独立$0-1$律}.
    \item 对于独立重复试验(每次发生的概率为$p$),有$\sum_{n=1}^{\infty}\mathbb{P}(A_n)\leq \infty$当且仅当$p=0$.也就是说,\textbf{在独立重复试验中,当事件几乎必然不发生时,无穷次地重复下去几乎必然不会有无穷次事件发生;当事件每次发生的概率不为0时,无论这个概率多么地小,只要无穷次地重复下去几乎必然会有无穷次事件发生.} 
    \item 对于独立试验(第$n$次发生的概率为$p_n(>0)$,无穷次地做下去),\textbf{事件列中是否有无穷多个事件发生只和$n$充分大时$p_n$的值有关}(这是符合直觉的).然而,直觉所感受不到的是:$p_n=\frac{1}{n}$时,事件列中会有无穷多个事件发生;$p_n=\frac{1}{n^2}$时,则不会有无穷多个事件发生.
    \item Borel-Canteli引理是强大数定律的基础(参见之后的强大数定律).
\end{enumerate}
\section{条件概率、全概率公式、Bayes公式}
\subsection{条件概率}
\textbf{条件概率的本质就是对样本空间的限制.}
\begin{definition}{条件概率}
\end{definition}
\subsection{全概率公式与Bayes公式}
\begin{theorem}{全概率公式}{}
    事件$\{A_i\}_{i=1}^{n}$互斥, $B\subset \cup_{i=1}^{n}A_i$, 则
    $$\mathbb{P}(B)=\sum_{i=1}^{n}\mathbb{P}(B\mid A_i)\mathbb{P}(A_i)$$
\end{theorem}
证明很简单,$B=\cup_{i=1}^{n}BA_i$再利用概率的可列可加性并用条件概率的乘法公式处理即可.

以下是两个常见的应用.
\begin{example}{(抽签公平性)}
    $n$个球,有$m$个黑球,剩下全为白,球除了颜色外没有任何差别. 求证:
    无放回地依次抽取球, 每一次抽中黑球的概率都是$\frac{m}{n}$.
\end{example}
再来一个更加有趣的:赌徒破产模型,同时在这里也会介绍一种在概率论中非常常见的处理方法:\textbf{递推公式法}.
\begin{example}{(赌徒破产模型)}{}
    一个人有$a$的本金, 打算再赢$b$元就停止赌博, 设每局$p=\frac{1}{2}$概率赢, 输赢对金钱的影响都是1, 输光后自然地停止赌博, 求输光的概率$p(a)$.
\end{example}
\subsection{独立性}
首先介绍事件的独立性. 对于两个事件独立, 也就是意味这一个事件的发生与否对另一个事件的发生与否没有任何影响, 很自然地, 
可以用条件概率来刻画这一性质:$\mathbb{P}(A\mid B)=\mathbb{P}(A)$, 但这里有一个细节需要处理, 条件概率要求$\mathbb{P}(B)>0$, 故而我们用乘法公式来改进定义.
\begin{definition}{两事件的独立性}{}
    对于事件$A,B$, 若$\mathbb{P}(AB)=\mathbb{P}(A)\mathbb{P}(B)$, 则称$A,B$独立.
\end{definition}
\begin{example}
    (分支过程). 设某种单性繁殖的生物群(如果是两性繁殖的生物,只考虑男性及其男性的后代)中每个个体进行独立繁衍, 每个个体产生$k$个下一代个体的概率为$p_k,k=0,1,2,\cdots$.
    记$m=\sum_{k=1}^{\infty}kp_k$. 设该生物群开始时(即第0代)只有一个个体. 证明:如果$m\leq 1,p_1<1$, 则这一生物群灭绝(即到某一代时个体数为0)的概率为1.
\end{example}
