\documentclass[lang=cn,10pt]{elegantbook}
\usepackage{ctex}
\title{中级宏观经济学(拔尖班)笔记}
\subtitle{2024-2025学年秋冬学期}

\author{Yishu Jiang}
\institute{School of Economics,Zhejiang University}
\date{\today}

\extrainfo{Talk is cheap,show me your solution!}

\setcounter{tocdepth}{3}

\logo{ZJU.png}
\cover{Cover.png}

% 本文档命令
\usepackage{array}
\newcommand{\ccr}[1]{\makecell{{\color{#1}\rule{1cm}{1cm}}}}

% 修改标题页的橙色带
% \definecolor{customcolor}{RGB}{32,178,170}
% \colorlet{coverlinecolor}{customcolor}

\setcounter{tocdepth}{2}
\begin{document}

\maketitle
\frontmatter
\chapter*{前言}
\markboth{Introduction}{前言}
{\fangsong 
    本笔记是2024-2025学年秋冬学期面向经济学拔尖班开设的中级宏观经济学的学习笔记,主要使用的教材为何樟勇《高级宏观经济学》和卢卡斯的《经济动态的递归方法》,授课老师为邬介然.

    由于本人初次学习中宏且水平能力有限,笔记中有所疏漏处,恳请指正.
}
\newpage

\tableofcontents

\mainmatter
\part{中宏的简单模型}
\chapter{单期代表性行为人模型}
本章介绍最简单的一类宏观经济模型:代表性行为人模型(Representative Agent Model),在这个模型中,抽象掉\textbf{异质性和分配问题}(例如收入、财富、偏好等都被同质化假设掉了).
假定经济由一个代表性的企业和一个代表性的消费者组成(这和有许多本质相同的企业和许多本质相同的消费者所组成的经济是相同的).

基本的分析思路是:首先关注行为人在约束条件下的最优化问题,然后分析不同行为人的自利行为交互会产生怎样的效果.首先我们需要知道消费者的偏好,企业的生产技术以及消费者与企业的资源禀赋.

{\kaishu 在正式开始分析前, 我们先讨论一个非常基本的问题: 为什么我们可以用代表性行为人的假设抽象掉所有的异质性?事实上, 它的理论基础是Arrow-Debru定理.当然,这里不深究具体的技术细节和定理证明,主要探讨思路上的问题:
在宏观的视角下,先不考虑个体偏好的差异,考虑更加易于度量的“收入和财富”的异质性,并试图抽象掉它——引入所谓的}“\textbf{个体冲击}(individual shock)”,{\kaishu 这里我们不管中间过程如何,把生下来到长大这段时间中的乱七八糟的都忽略掉, 把人与人之间的收入和财富差异都抽象成一种外部冲击: 即每个人在本质上是一样的, 但由于存在外部冲击, 
导致了收入和财富的不同, 继而使得每个消费者的最优化选择出现差异;而在总体上, 我们希望通过某种“内部交易”方式, 让这些内部冲击相互对冲掉又不会对总体性的结果造成影响.于是我们引入“保险”来对冲风险, 假设所有人初始财富一致, 彼此之间通过互保的方式应对外部风险, 那么, }\textbf{假定金融市场是完备的(这是很理想的假设,现实中并做不到, 因为现实中的保险交易次数是可数的而风险可以连续变动)},{\kaishu 无穷期的交易下去, 最终的所有人的财富都应当是一样的(因为把彼此之间的个体冲击都对冲掉了, 相当于没有个体冲击).
这就是Arrow-Debru定理的基本思想, 在此基础上他们证明了福利经济学基本定理, 这一思想也是代表性行为人模型的理论基础.} 
\section{决策环境:偏好、技术与禀赋}
\subsection{基本环境}
\begin{enumerate}
    \item 经济活动只进行\textbf{一期}.
    \item 经济中只有一个代表性消费者和一个代表性企业(相当于有许多本质上相同的消费者和企业放在一块儿).
    \item 代表性消费者决定最优的资本供给和劳动供给;代表性企业决定最优的资本和劳动使用量.
    \item 消费者和企业的行为都是\textbf{完全竞争}的(\textbf{对消费者和企业而言, 价格是外生给定的}).
    \item 消费者拥有企业(因为在均衡的时候, 市场出清, 相当于消费者提供的劳动和资本都转化为了产品并被消费完全).
\end{enumerate}
需要注意的是, 这些假设都是基于均衡性质的结果, 经济模型本身是动态的, 但在这里我们只分析经济达到均衡时的情形, 因而可以使用上述假设, 在动态模型的过程中不能这么干;
类似地, 在处理总的消费的时候也需要注意个体加总的问题, 总消费函数与个体消费函数的含义有根本上的不同, 不能出现混为一谈的问题.
\subsection{偏好}
我们用效用函数来表示消费者的偏好:
$$u(c,l)$$
其中$c$表示消费,$l$是闲暇. 两者之间存在一个替代作用:每天的时间是固定的,闲暇时间越长,劳动时间越短,收入就越少,消费也就越少,反之同理.
效用函数满足以下假设:
\begin{enumerate}
   \item 严格凹(确保最优解存在)、二次可微;
   \item 严格增:$u(\cdot)$对$c$和$l$增;
   \item Inada条件(规范性条件,确保没有角点解):$$\frac{\partial u}{\partial c}(0,l)=+\infty,\frac{\partial u}{\partial c}(+\infty,l)=0;\frac{\partial u}{\partial l}(c,0)=+\infty,\frac{\partial u}{\partial l}(c,+\infty)=0$$
\end{enumerate}
\subsection{技术}
我们用生产函数刻画技术:
$$y=zf(k,n)$$
其中$k$表示资本要素投入, $n$表示劳动要素投入, $y$表示产出, $z$是全要素生产参数(TFP, 用于解释在给定相同的资本和劳动的情形下产出的不同, 反应劳动和资本的效率).
$f$满足如下假设:
\begin{enumerate}
    \item 严格拟凹(生产问题的形式是无约束的最优化,不需要用到更强的“严格凹”假设)、二次可微;
    \item 严格增;
    \item 一次齐次(\textbf{表明规模报酬不变});
    \item Inada条件
\end{enumerate}
\subsection{禀赋}
消费者拥有$k_0$的初始资本, 初始资本只能租借给企业不能用于自己消费, 资本与消费品可以转换; 消费者还拥有$h$的时间禀赋, 可以用于劳动也可以用于闲暇.
\newpage

\chapter{两期动态模型}
\newpage

\chapter{三期动态模型}
\newpage

\chapter{无穷期动态模型}
\newpage

\part{数学工具}
\chapter{动态规划Dynamic Programing}

\chapter{基本的数值计算方法}

\part{DEGE:动态随机一般均衡}
\chapter{真实经济周期模型RBC}

\chapter{Banking Model}

\chapter{OLG模型}

\chapter{金融摩擦}

\part{附录}
\chapter{数学工具}
\section{哈密尔顿方法}
\chapter{宏观经济学的微观经济理论基础}
\section{Arrow-Debru定理}
\chapter{读书笔记:《宏观经济学史:从凯恩斯到卢卡斯及其后》}
\end{document}
