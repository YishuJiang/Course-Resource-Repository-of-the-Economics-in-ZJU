\documentclass[lang=cn,10pt]{elegantbook}
\usepackage{ctex}
\title{常微分方程笔记}
\subtitle{2024-2025学年秋冬学期}

\author{Yishu Jiang}
\institute{School of Economics,Zhejiang University}
\date{\today}

\extrainfo{Talk is cheap,show me your solution!}

\setcounter{tocdepth}{3}

\logo{ZJU.png}
\cover{Cover.png}

% 本文档命令
\usepackage{array}
\newcommand{\ccr}[1]{\makecell{{\color{#1}\rule{1cm}{1cm}}}}

% 修改标题页的橙色带
% \definecolor{customcolor}{RGB}{32,178,170}
% \colorlet{coverlinecolor}{customcolor}

\setcounter{tocdepth}{2}
\begin{document}

\maketitle
\frontmatter
\chapter*{前言}
\markboth{Introduction}{前言}
{\fangsong 
    本笔记是2024-2025学年秋冬学期3.5学分概率论的学习笔记,基于周青龙老师的板书并参考中科大赵立丰老师、清华大学刘思齐老师的常微分方程课程整理而成.

    由于本人初次学习概率论且水平能力有限,笔记中有所疏漏处,恳请指正.
}
\newpage

\tableofcontents

\mainmatter
\chapter{常微分方程的解法}
\section{恰当方程}
\section{一阶线性方程}
\section{一阶隐式方程}
\section{其他重要的常微分方程}
\section{幂级数解法}
\newpage

\chapter{解的存在性和唯一性}
\section{Picard定理}
\section{解的延伸定理}
\section{比较定理}
\newpage

\chapter{解对初值和参数的依赖性}
\section{解对初值和参数的连续依赖性}
\section{解对初值和参数的连续可微性}
\newpage

\chapter{线性微分方程组}
\section{线性微分方程组的一般理论}
\section{常系数线性微分方程组}
\section{高阶线性微分方程组}
\section{周期系数线性微分方程组}
\newpage

\chapter{边值问题}
\section{Sturm比较定理}
\section{Strum-Liouville边值问题}
\section{特征函数系的正交性}
\section{周期边值问题}
\newpage

\chapter{常微分方程定性理论}
\section{动力系统、相空间、轨线}
\section{Lyapunov稳定性}
\section{平面奇点和极限环}
\newpage

\chapter{一阶偏微分方程}
\section{首次积分}

\end{document}
